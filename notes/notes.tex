\documentclass{article}

% For math environments
\usepackage{amsmath, amsfonts}
% For links
\usepackage[colorlinks=true,
    linkcolor = blue,
    urlcolor  = blue,
    citecolor = blue,
    anchorcolor = blue]{hyperref}
% So spaces are put between paragraphs
\usepackage{parskip}
% For figures
\usepackage{tikz}
% Set the margins to not be ridiculous
\usepackage[margin=0.75in]{geometry}
% For code blocks
\usepackage{listings}

% Make listings look better
\lstset{
  basicstyle=\ttfamily,
  columns=fullflexible,
  keepspaces=true,
}

% Useful commands
\newcommand\codebox[2]{
    \begin{center}
        \framebox[#1][l]{
            \lstinputlisting{#2}
        }
    \end{center}
}

% Monospace inline code
\def\code#1{\texttt{#1}}

% Commands that make life easier
\newcommand\gath[1]{\begin{gather} #1 \end{gather}}
\newcommand\gaths[1]{\begin{gather*} #1 \end{gather*}}
\newcommand\ali[1]{\begin{align} #1 \end{align}}
\newcommand\parens[1]{\left( #1 \right)}
\newcommand\squares[1]{\left[ #1 \right]}
\newcommand\braces[1]{\left\{ #1 \right\}}
\newcommand\angles[1]{\left\langle #1 \right\rangle}
\newcommand\deriv[2]{\frac{d #1}{d #2}}
\newcommand\abs[1]{\left| #1 \right|}
\newcommand\floor[1]{\left\lfloor #1 \right\rfloor}
\DeclareMathOperator{\lcm}{lcm}
\def\non{\nonumber \\}

% Common sets
\def\ints{{\mathbb{Z}}}
\def\pints{{\ints_+}}
\def\nints{{\ints_-}}
\def\rats{{\mathbb{Q}}}
\def\reals{{\mathbb{R}}}
\def\realsnz{\reals - \braces{0}}
\def\realsp{\reals_+}
\def\preals{{\reals_+}}
\def\cpx{\mathbb{C}}
\def\rinf{\reals^\infty}
\def\es{\varnothing}
\newcommand\intsfin[1]{\braces{1, \ldots, #1}}
\newcommand\cpfin[2]{#1_1 \times \cdots \times #1_{#2}}

% Commands for problems
\newcommand\notes[4]{
    \section*{#1 Day #2}
    \textbf{Puzzle:} \href{#3}{#3}

    #4
}


\begin{document}

\title{Advent of Code Solution Notes}
\author{Dan Whitman}
\date{}

\maketitle

\notes{2021}{24}{https://adventofcode.com/2021/day/24}{
    This problem involved quite a bit of analytical work to reverse engineer the MONAD program (i.e. the puzzle input).
    This was necessary to determine the properties of valid model numbers since the search space of the $14$-digit model numbers is too large to exhaustively search.
    First, MONAD can be divided into $14$ sections that deal with each digit of the model number, and the sections are largely identical to each other except for some key parameters.
    The code for each section has the form:

    \codebox{3in}{2021-24-monad-template.txt}

    Each MONAD section has the three parameters $a_n$, $b_n$, and $c_n$, where we note that always $a_n \in \braces{1, 26}$.
    The actual parameters of each section are
    \begin{center}
        \begin{tabular}{c|ccc|c}
            $n$ & $a_n$ & $b_n$ & $c_n$ & Kind       \\
            \hline
            1   & 1     & 11    & 3     & Increasing \\
            2   & 1     & 14    & 7     & Increasing \\
            3   & 1     & 13    & 1     & Increasing \\
            4   & 26    & -4    & 6                  \\
            5   & 1     & 11    & 14    & Increasing \\
            6   & 1     & 10    & 7     & Increasing \\
            7   & 26    & -4    & 9                  \\
            8   & 26    & -12   & 9                  \\
            9   & 1     & 10    & 6     & Increasing \\
            10  & 26    & -11   & 4                  \\
            11  & 1     & 12    & 0     & Increasing \\
            12  & 26    & -1    & 7                  \\
            13  & 26    & 0     & 12                 \\
            14  & 26    & -11   & 1                  \\
        \end{tabular}
    \end{center}


    After the $n$th section let $z_n$ be the value in the \code{z} register at the end of each section, where of course initially $z_0 = 0$.
    Also let $d_n$ be the $n$th digit so that $1 \leq d_n \leq 9$.
    Then the above code for each section results in the following:
    \gath{
        z_n = \begin{cases}
            \floor{\frac{z_{n-1}}{a_n}}                & d_n = (z_{n-1} \mod 26) + b_n    \\
            26 \floor{\frac{z_{n-1}}{a_n}} + d_n + c_n & d_n \neq (z_{n-1} \mod 26) + b_n \\
        \end{cases} \label{eqn:2021:24:f}
    }
    First, note that, for all the sections for which $a_n = 1$, we have that $b_n > 9$ so that clearly $(z_{n-1} \mod 26) + b_n > 9$, and hence $d_n \neq (z_{n-1} \mod 26) + b_n$.
    From this it follows that
    \gath{
        z_n = 26 \floor{\frac{z_{n-1}}{a_n}} + d_n + c_n = 26 z_{n-1} + d_n + c_n
    }
    since $a_n = 1$.
    It will obviously be the case here that $z_n > z_{n-1}$ and, for this reason, these sections are marked as ``Increasing'' in the table above.

    Since a valid model number must result in $z_{14} = 0$ and half of the sections will increase $z$, let us assume that the other seven sections must \emph{decrease} $z$.
    In particular, we suppose that all of these sections (viz. for $n \in \braces{4, 7, 8, 10, 12, 13, 14}$) apply the first case of \eqref{eqn:2021:24:f}, which imposes restrictions on these digits.
    Now, the first three sections result in the following:
    \ali{
        z_1 &= 26 z_0 + d_1 + c_1 \non
        &= d_1 + 3 \\
        z_2 &= 26 z_1 + d_2 + c_2 = 26(d_1 + 3) + d_2 + c_2 \non
        &= 26 d_1 + d_2 + 85 \\
        z_3 &= 26 z_2 + d_3 + c_3 = 26(26 d_1 + d_2 + 85) + d_3 + 1 \non
        &= 676d_1 + 26d_2 + d_3 + 2211.
    }
    Regarding $d_4$ and $z_4$, define functions
    \ali{
        f(z, b) &= (z \mod 26) + b \\
        g(z) &= \floor{\frac{z}{26}}
    }
    for brevity.
    Then, since we assume that $z_4$ is determined from the first case of \eqref{eqn:2021:24:f}, it must be that
    \ali{
        d_4 = f(z_3, -4) \\
        z_4 = g(z_3).
    }
    Since the $n \in \braces{5, 6}$ sections again increase $z$, we have
    \ali{
        z_5 &= 26 z_4 + d_5 + c_5 \non
        &= 26 z_4 + d_5 + 14 \\
        z_6 &= 26 z_5 + d_6 + c_6 = 26(26 z_4 + d_5 + 14) + d_6 + 7 \non
        &= 676 z_4 + 26 d_5 + d_6 + 371.
    }
    Continuing in this way for both the increasing and decreasing sections gives
    \ali{
        d_7 &= f(z_6, -4) & z_{11} &= 26 z_{10} + d_{11} + c_{11} \\
        z_7 &= g(z_6) & &= 26 z_{10} + d_{11} \\
        d_8 &= f(z_7, -12) & d_{12} &= f(z_{11}, -1) \\
        z_8 &= g(z_7) & z_{12} &= g(z_{11}) \\
        z_9 &= 26 z_8 + d_9 + c_9 & d_{13} &= f(z_{12}, 0) \\
        &= 26 z_8 + d_9 + 6 & z_{13} &= g(z_{12})  \\
        d_{10} &= f(z_9, -11) & d_{14} &= f(z_{13}, -11) \\
        z_{10} &= g(z_9) & z_{14} &= g(z_{13})
    }

    Thus, we are free to choose the digits $d_n$ for $n \in \braces{1, 2, 3, 5, 6, 9, 11}$ whereas those for $n \in \braces{4, 7, 8, 10, 12, 13, 14}$ are all determined based on those we choose.
    This massively reduces our required search space from $9^{14}$ possible model numbers to only $9^7$ model numbers, which is much more feasible to run in a reasonable amount of time.
    So, in our program, we search every possible combination of the appropriate digits, that is
    \gath{
        (d_1, d_2, d_3, d_5, d_6, d_9, d_{11}) \in D \times D \times D \times D \times D \times D \times D,
    }
    where $D = \braces{1, 2, 3, 4, 5, 6, 7, 8, 9}$ is the set of valid digits.
    For the first part we start each digit at $9$ then move backwards to $1$ since we are looking for the \emph{largest} valid model number, for the second part we do the opposite since we are looking for the \emph{smallest} model number.

    For each model number digits, we calculate $z_n$ and/or $d_n$ in sequence for increasing $n$ using the equations above.
    If any of the digits $d_n$ are not in the set $D$ we discard the model number, and similarly if we finally calculate $z_{14}$, and it is nonzero.
    Given our search order, the first model number that meets all of these conditions is our answer in either part.
    Finally, the first valid model digits are passed as input to the interpreted MONAD program to verify its validity.
}

\end{document}
