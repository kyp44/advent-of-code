\documentclass{article}

% For math environments
\usepackage{amsmath, amsfonts}
% For links
\usepackage[colorlinks=true,
    linkcolor = blue,
    urlcolor  = blue,
    citecolor = blue,
    anchorcolor = blue]{hyperref}
% So spaces are put between paragraphs
\usepackage{parskip}
% For figures
\usepackage{tikz}
% Set the margins to not be ridiculous
\usepackage[margin=0.75in]{geometry}
% For code blocks
\usepackage{listings}

% Make listings look better
\lstset{
  basicstyle=\ttfamily,
  columns=fullflexible,
  keepspaces=true,
}

% Useful commands
\newcommand\codebox[2]{
    \begin{center}
        \framebox[#1][l]{
            \lstinputlisting{#2}
        }
    \end{center}
}

% Monospace inline code
\def\code#1{\texttt{#1}}

% Commands that make life easier
\newcommand\gath[1]{\begin{gather} #1 \end{gather}}
\newcommand\gaths[1]{\begin{gather*} #1 \end{gather*}}
\newcommand\ali[1]{\begin{align} #1 \end{align}}
\newcommand\parens[1]{\left( #1 \right)}
\newcommand\squares[1]{\left[ #1 \right]}
\newcommand\braces[1]{\left\{ #1 \right\}}
\newcommand\angles[1]{\left\langle #1 \right\rangle}
\newcommand\deriv[2]{\frac{d #1}{d #2}}
\newcommand\abs[1]{\left| #1 \right|}
\newcommand\floor[1]{\left\lfloor #1 \right\rfloor}
\DeclareMathOperator{\lcm}{lcm}
\def\non{\nonumber \\}

% Commands for problems
\newcommand\notes[4]{
    \section*{#1 Day #2}
    \textbf{Puzzle:} \href{#3}{#3}

    #4
}


\begin{document}

\title{Advent of Code Solution Notes}
\author{Dan Whitman}
\date{}

\maketitle

\notes{2015}{25}{https://adventofcode.com/2015/day/25}{
    As part of this problem we have the following infinite table in which the order is filled in along diagonals:
    \begin{center}
        \begin{tabular}{c|ccccccc}
                     & 1        & 2        & 3        & 4        & 5        & 6        & $\cdots$ \\
            \hline
            1        & 1        & 3        & 6        & 10       & 15       & 21       & $\cdots$ \\
            2        & 2        & 5        & 9        & 14       & 20       & 27       & $\cdots$ \\
            3        & 4        & 8        & 13       & 19       & 26       & 34       & $\cdots$ \\
            4        & 7        & 12       & 18       & 25       & 33       & 42       & $\cdots$ \\
            5        & 11       & 17       & 24       & 32       & 41       & 51       & $\cdots$ \\
            6        & 16       & 23       & 31       & 40       & 50       & 61       & $\cdots$ \\
            $\vdots$ & $\vdots$ & $\vdots$ & $\vdots$ & $\vdots$ & $\vdots$ & $\vdots$ & $\ddots$ \\
        \end{tabular}
    \end{center}
    To solve the problem, for a given row and column number, we need to determine which number in the above table is at those coordinates.
    Given the diagonal pattern, this can be done analytically.
    Let $m$ denote a row and $n$ denote a column, where of course $n, m \geq 1$, and let $f(m, n)$ denote the number at the position.

    Now, we notice a pattern in the first row in which the $n$th number is the previous number plus $n$.
    This results in the following:
    \gath{
        f(1, n) = \sum_{j=1}^n j = \frac{n(n + 1)}{2},
    }
    which are the triangular numbers.
    The $n$th column also follows a similar pattern in which the $m$th number down is the previous number plus $n + m - 2$.
    This results in the following general formula we seek:
    \ali{
        f(m, n) &= f(1, n) + \sum_{i=0}^{m-2}(n + i) = f(1, n) + \sum_{i=1}^{m-1}(n + i - 1) \non
        &= \frac{n(n + 1)}{2} + \sum_{i=1}^{m-1} (n - 1) + \sum_{i=1}^{m-1} i = \frac{n(n + 1)}{2} + (m - 1)(n - 1) + \frac{m(m - 1)}{2} \non
        &= \frac{n(n + 1) + m(m - 1) + 2(n - 1)(m - 1)}{2} = \frac{n^2 + n + m^2 - m + 2nm - 2m - 2n + 2}{2} \non
        &= \frac{n^2 + m^2 + 2nm - n - 3m + 2}{2} \non
        &= \frac{(n + m)^2 - n -3m + 2}{2}.
    }
    Note that we have followed the usual convention that $\sum_{i=a}^b = 0$ when $b < a$, and also that clearly
    \gath{
        \sum_{i=1}^{m-1} i = \frac{m(m - 1)}{2} = 0
    }
    when $m = 1$ as we would like according to this convention.
}

\notes{2021}{8}{https://adventofcode.com/2021/day/8}{
    \def\WW{\mathbb{W}}
    \def\SS{\mathbb{S}}
    There are a total of seven segments, namely $\braces{a, b, c, d, e, f, g}$.
    For the digit $0 \leq n \leq 9$, let $S_n$ be the set of segments used to create the digit $n$, and let $\SS = \braces{S_n \mid 0 \leq n \leq 9}$ be the set of these digit sets.
    Therefore, from what we are given, we have
    \ali{
        S_0 &= \braces{a, b, c, e, f, g} & S_5 &= \braces{a, b, d, f, g} \non
        S_1 &= \braces{c, f} & S_6 &= \braces{a, b, d, e, f, g} \non
        S_2 &= \braces{a, c, d, e, g} & S_7 &= \braces{a, c, f} \non
        S_3 &= \braces{a, c, d, f, g} & S_8 &= \braces{a, b, c, d, e, f, g} \non
        S_4 &= \braces{b, c, d, f} & S_9 &= \braces{a, b, c, d, f, g} \nonumber
    }
    For a given entry, we must find the one-to-one map of the wire letters to the segment letters, as this allows us to determine which digits are being displayed.
    Note that this is only needed for part two of the problem.
    Denote these wire letters with $w_x$ where $x \in \braces{a, b, c, d, e, f, g}$ so that $w_x$ maps to segment $x$.
    For the entry, we are given a set of sets $\WW = \braces{W_n \mid 0 \leq n \leq 9}$, where each $W_n$ corresponds to $S_n$ and contains exactly those wires that correspond to the segments in $S_n$.
    So, for example, $W_7 = \braces{w_a, w_c, w_f}$.
    However, we initially do not know which $W_n \in \WW$ is which.

    The following algorithm was developed by experimentation with the sets in Python, and is implemented by the code to solve the problem.
    First, we \emph{do} know how many elements each $W_n \in \WW$ has, and as discussed in part one of the problem description, some sets $S_n \in \SS$ have a unique number of elements.
    In particular, $S_1$ is the only set with only two elements, $S_4$ is the only set with four elements, $S_7$ is the only set with three elements, and $S_8$ is the only set with all seven elements.
    All the other sets have either five or six elements.
    Hence, we know right away exactly which sets in $\WW$ are $W_1$, $W_4$, $W_7$, and $W_8$.

    Next, it was noticed that $S_7 - S_1 = \braces{a}$, and so
    \gath{
        \braces{w_a} = W_7 - W_1. \label{eqn:2021:8:wa}
    }
    Now note that $S_2$, $S_3$, and $S_5$ are the sets in $\SS$ with five elements.
    Similarly, $S_0$, $S_6$, and $S_9$ are the sets in $\SS$ with six elements.
    So let
    \gath{
        J_5 = \bigcap \braces{S_2, S_3, S_5} = \braces{a, d, g} \\
        J_6 = \bigcap \braces{S_0, S_6, S_9} = \braces{a, b, f, g}
    }
    and, analogously,
    \gath{
        I_5 = \bigcap \braces{W \in \WW \mid \abs{W} = 5} \\
        I_6 = \bigcap \braces{W \in \WW \mid \abs{W} = 6}
    }
    so that $I_5$ and $I_6$ are fully defined from our given entry data since we know how many elements each $W \in \WW$ has.
    It then follows that $J_5 \cap J_6 = \braces{a, g}$, and hence
    \gath{
        \braces{w_g} = \parens{I_5 \cap I_6} - \braces{w_a}
    }
    since we have already determined $w_a$ in \eqref{eqn:2021:8:wa}.
    Next we have that clearly $J_5 - \braces{a, g} = \braces{d}$ so that
    \gath{
        \braces{w_d} = I_5 - \braces{w_a, w_g}.
    }
    At this point we have determined three of the seven wire mappings.

    The next step is to notice that $J_6 \cap S_1 = \braces{f}$, and thus
    \gath{
        \braces{w_f} = I_6 \cap W_1.
    }
    From this it immediately follows that $S_1 - \braces{f} = \braces{c}$ so that
    \gath{
        \braces{w_c} = W_1 - \braces{w_f}.
    }
    The remaining two mappings are easily found at this point as we first have $S_4 - \braces{c, d, f} = \braces{b}$ so that
    \gath{
        \braces{w_b} = W_4 - \braces{w_c, w_d, w_f},
    }
    and $S_8 - \braces{a, b, c, d, f, g} = \braces{e}$ so that
    \gath{
        \braces{w_e} = W_8 - \braces{w_a, w_b, w_c, w_d, w_f, w_g}.
    }
    At this point we know the complete mapping from wires to segments and so can determine which digits are displayed.
}

\notes{2021}{17}{https://adventofcode.com/2021/day/17}{
    \def\vyo{v_{y0}}
    \def\vxo{v_{x0}}

    This problem involves simulating probe trajectories for many initial velocities to see whether it hits the target.
    It is of course easy and computationally inexpensive to simulate trajectories, but we must bound the problem, so that we have a finite set of initial velocities to simulate.

    First let us focus on the $y$ dimension.
    We assume that $x$ is constant and within the target so that we can determine the possible range of initial $y$ velocities $\vyo$ for which it is possible to hit the target regardless of the initial $x$ velocity $\vxo$.
    Denote the discrete steps by integers $n$, so that at $n = 0$ is when the probe is at its initial position of $(0, 0)$.
    Let $v_y(n)$ and $y(n)$ the be the probe velocity and position in $y$ at step $n$.
    By what is given, we then clearly have the following recursive relationships:
    \ali{
        v_y(0) &= \vyo & y(0) &= 0 \label{eqn:2021:17:yinit} \\
        v_y(n) &= v_y(n-1) - 1 & y(n) &= y(n-1) + v_y(n-1). \label{eqn:2021:17:yrec}
    }
    It was then reasoned that, in general,
    \gath{
        v_y(n) = \vyo - n, \label{eqn:2021:17:vyn}
    }
    which is easy to prove by induction.
    First, clearly \eqref{eqn:2021:17:vyn} is true for $n = 0$ from \eqref{eqn:2021:17:yinit}.
    Next, assume that \eqref{eqn:2021:17:vyn} is true for $n$ so that we have
    \gath{
        v_y(n+1) = v_y(n) - 1 = \vyo - n - 1 = \vyo - (n+1),
    }
    which proves that \eqref{eqn:2021:17:vyn} holds for $n+1$ as well, completing the induction step.

    Similarly, it was also reasoned that
    \gath{
        y(n) = n\vyo - \sum_{i=1}^{n-1} i = n\vyo - \frac{(n-1)n}{2} \label{eqn:2021:17:yn}
    }
    in general.
    This is also easy to prove by induction.
    As in \eqref{eqn:2021:17:yinit}, clearly $y(0) = 0$ by \eqref{eqn:2021:17:yn} as well, showing the base case.
    Now assume that \eqref{eqn:2021:17:yn} holds for $n$ so that, by \eqref{eqn:2021:17:yrec}, we have
    \ali{
        y(n+1) &= y(n) + v_y(n) = \squares{n\vyo - \frac{(n-1)n}{2}} + \parens{\vyo - n} \non
        &= (n+1) \vyo - \squares{\frac{(n-1)n}{2} + n} = (n+1)\vyo - \frac{n^2 - n + 2n}{2} \non
        &= (n+1) \vyo - \frac{n^2+n}{2} = (n+1) \vyo - \frac{n(n+1)}{2},
    }
    which shows the induction step since this is just $y(n+1)$ by \eqref{eqn:2021:17:yn}.

    Let the target zone (with respect to $y$) be defined by $a_y$ and $b_y$ where the probe is in the zone only when $a_y \leq y \leq b_y$.
    We note a couple of properties of which we take advantage:
    \begin{enumerate}
        \item It is the case that $a_y < b_y < 0$ for both the example and the actual problem inputs.
        \item Once the probe is entirely below the target zone, that is once $y(n) < a_y$, the probe will only fall further down forever (or until it hits the ocean floor as the case may be) and can never hit the target after this.
              This provides a terminating condition for the simulation of trajectories for a given $\vyo$.
              This is true since (by 1) the target is below the starting position so that the $v_y(n)$ must already be negative by the time the target is encountered.
    \end{enumerate}
    Observation 1 provides a lower bound on possible values of $\vyo$.
    For, if the probe is shot straight down with some $\vyo < 0$, then clearly, after a single step, $y(1) = \vyo$.
    As this cannot be below the target, it must be that
    \gath{
        \vyo = y(1) \geq a_y.
    }

    The probe can of course be launched up and then fall back down into the target, so we assume that the upper bound is positive.
    Hence, in what follows, assume that $\vyo > 0$.
    As what goes up must come down, let us see when the probe will fall back down to its starting depth, that is when
    \gath{
        y(n) =  n \vyo - \frac{(n-1)n}{2} = 0 \non
        n\vyo = \frac{(n-1)n}{2}.
    }
    Clearly this is true when $n = 0$, which is of course our starting position.
    As we are interested in other solutions, let $n \neq 0$ so that we can divide both sides by $n$:
    \gath{
        \vyo = \frac{n-1}{2} \non
        n = 2\vyo + 1.
    }
    Thus, at this step, the probe will reach our starting position again.
    At the next step the position will of course be
    \ali{
        y(2\vyo + 2) &= (2\vyo + 2)\vyo - \frac{(2\vyo + 1)(2\vyo + 2)}{2} = 2\vyo^2 + 2\vyo - \frac{4\vyo^2 + 6\vyo + 2}{2} \non
        &= 2\vyo^2 + 2\vyo - 2 \vyo^2 - 3 \vyo - 1 = -(\vyo + 1)
    }
    Clearly if this position is below the target zone, then the probe will have completely missed the target since it was above it (at zero) in the previous step!
    Thus, it must be that
    \gath{
        y(2\vyo + 2) = -(\vyo + 1) \geq a_y \non
        \vyo + 1 \leq -a_y \non
        \vyo \leq -(a_y + 1),
    }
    which is of course an upper bound on $\vyo$.
    Thus, we need consider only
    \gath{
        \vyo \in I_y = \braces{k \in \ints \mid a_y \leq k \leq -(a_y + 1)},
    }
    noting that of course $I_y$ is a finite set.

    Lastly, we note that, just like in continuous calculus, if the probe is launched up then the maximum position will be that when $v_y(n) = 0$ since, after this the velocity will become negative, and it will start falling back down.
    This of course occurs when
    \gath{
        v_y(n) = \vyo - n = 0 \non
        n = \vyo.
    }
    Hence, the maximum position in this case is
    \gath{
        y(\vyo) = \vyo^2 - \frac{(\vyo - 1)\vyo}{2} = \frac{2 \vyo^2 - \vyo^2 + \vyo}{2} = \frac{1}{2}\vyo(\vyo + 1).
    }
    Now, if the probe is just let go or launched downward, then clearly it will never go above its starting position.
    Therefore, the universal maximum $y$ position is
    \gath{
        y_\mathrm{max} = \begin{cases}
            0                         & \vyo \leq 0 \\
            \frac{1}{2}\vyo(\vyo + 1) & \vyo > 0.
        \end{cases}
    }

    Turning our attention to the $x$ dimension, again define the target area by $a_x$ and $b_x$ where the probe can only be in the area when $a_x \leq x(n) \leq b_x$.
    We note the following observations:
    \begin{enumerate}
        \item In both the example and actual input, the target area is to the right of the starting position, that is $0 < a_x < b_x$.
              Because of this, we need only consider positive initial velocities so that $\vxo > 0$.
        \item This situation is directly analogous to the $y$ situation when the probe is launched up in that it moves upward (i.e. rightward here), slowing down until the velocity reaches zero.
              At this point the situations diverge as the in the $y$ dimension the probe then begins falling downward, whereas in the $x$ dimension it does not now ``fall back'' (i.e. begin moving to the left), but instead stays where it is once the velocity reaches zero.
              Because of this the probe can never move left if it is thrown right (and vice versa in fact) so that its position is non-decreasing.
    \end{enumerate}
    Analogously to the $y$ dimension, these observations provide a simple upper bound on $\vxo$.
    For, if the probe is launched with such a $\vxo$ such that is completely bypasses the target area in a single step, then it can never return to it.
    Therefore,
    \gath{
        \vxo \leq b_x.
    }
    For the lower bound we can simply use 1 since we know the probe must be launched with a positive velocity.
    By considering the $x$ position where the probe will ultimately stop moving in $x$ (namely that this must not be to the left of the entire target zone), this bound can be tightened, but the result is messy, and using 1 instead has a negligible effect on the computation time.
    Thus, it must be that
    \gath{
        \vxo \in I_x = \braces{k \in \ints \mid 1 \leq k \leq b_x},
    }
    noting that again $I_x$ is clearly finite.
    Therefore, $I_x \times I_y$ is our finite set of initial velocities such that any initial velocity outside this set is guaranteed to never hit the target zone.
    However, not every $(\vxo, \vyo) \in I_x \times I_y$ will necessarily hit the target, so these trajectories must all be simulated to see which do and which do not.
}

\notes{2021}{24}{https://adventofcode.com/2021/day/24}{
    This problem involved quite a bit of analytical work to reverse engineer the MONAD program (i.e. the puzzle input).
    This was necessary to determine the properties of valid model numbers since the search space of the $14$-digit model numbers is too large to exhaustively search.
    First, MONAD can be divided into $14$ sections that deal with each digit of the model number, and the sections are largely identical to each other except for some key parameters.
    The code for each section has the form:

    \codebox{3in}{2021-24-monad-template.txt}

    Each MONAD section has the three parameters $a_n$, $b_n$, and $c_n$, where we note that always $a_n \in \braces{1, 26}$.
    The actual parameters of each section are
    \begin{center}
        \begin{tabular}{c|ccc|c}
            $n$ & $a_n$ & $b_n$ & $c_n$ & Kind       \\
            \hline
            1   & 1     & 11    & 3     & Increasing \\
            2   & 1     & 14    & 7     & Increasing \\
            3   & 1     & 13    & 1     & Increasing \\
            4   & 26    & -4    & 6                  \\
            5   & 1     & 11    & 14    & Increasing \\
            6   & 1     & 10    & 7     & Increasing \\
            7   & 26    & -4    & 9                  \\
            8   & 26    & -12   & 9                  \\
            9   & 1     & 10    & 6     & Increasing \\
            10  & 26    & -11   & 4                  \\
            11  & 1     & 12    & 0     & Increasing \\
            12  & 26    & -1    & 7                  \\
            13  & 26    & 0     & 12                 \\
            14  & 26    & -11   & 1                  \\
        \end{tabular}
    \end{center}


    After the $n$th section let $z_n$ be the value in the \code{z} register at the end of each section, where of course initially $z_0 = 0$.
    Also let $d_n$ be the $n$th digit so that $1 \leq d_n \leq 9$.
    Then the above code for each section results in the following:
    \gath{
        z_n = \begin{cases}
            \floor{\frac{z_{n-1}}{a_n}}                & d_n = (z_{n-1} \mod 26) + b_n    \\
            26 \floor{\frac{z_{n-1}}{a_n}} + d_n + c_n & d_n \neq (z_{n-1} \mod 26) + b_n \\
        \end{cases} \label{eqn:2021:24:f}
    }
    First, note that, for all the sections for which $a_n = 1$, we have that $b_n > 9$ so that clearly $(z_{n-1} \mod 26) + b_n > 9$, and hence $d_n \neq (z_{n-1} \mod 26) + b_n$.
    From this it follows that
    \gath{
        z_n = 26 \floor{\frac{z_{n-1}}{a_n}} + d_n + c_n = 26 z_{n-1} + d_n + c_n
    }
    since $a_n = 1$.
    It will obviously be the case here that $z_n > z_{n-1}$ and, for this reason, these sections are marked as ``Increasing'' in the table above.

    Since a valid model number must result in $z_{14} = 0$ and half of the sections will increase $z$, let us assume that the other seven sections must \emph{decrease} $z$.
    In particular, we suppose that all of these sections (viz. for $n \in \braces{4, 7, 8, 10, 12, 13, 14}$) apply the first case of \eqref{eqn:2021:24:f}, which imposes restrictions on these digits.
    Now, the first three sections result in the following:
    \ali{
        z_1 &= 26 z_0 + d_1 + c_1 \non
        &= d_1 + 3 \\
        z_2 &= 26 z_1 + d_2 + c_2 = 26(d_1 + 3) + d_2 + c_2 \non
        &= 26 d_1 + d_2 + 85 \\
        z_3 &= 26 z_2 + d_3 + c_3 = 26(26 d_1 + d_2 + 85) + d_3 + 1 \non
        &= 676d_1 + 26d_2 + d_3 + 2211.
    }
    Regarding $d_4$ and $z_4$, define functions
    \ali{
        f(z, b) &= (z \mod 26) + b \\
        g(z) &= \floor{\frac{z}{26}}
    }
    for brevity.
    Then, since we assume that $z_4$ is determined from the first case of \eqref{eqn:2021:24:f}, it must be that
    \ali{
        d_4 = f(z_3, -4) \\
        z_4 = g(z_3).
    }
    Since the $n \in \braces{5, 6}$ sections again increase $z$, we have
    \ali{
        z_5 &= 26 z_4 + d_5 + c_5 \non
        &= 26 z_4 + d_5 + 14 \\
        z_6 &= 26 z_5 + d_6 + c_6 = 26(26 z_4 + d_5 + 14) + d_6 + 7 \non
        &= 676 z_4 + 26 d_5 + d_6 + 371.
    }
    Continuing in this way for both the increasing and decreasing sections gives
    \ali{
        d_7 &= f(z_6, -4) & z_{11} &= 26 z_{10} + d_{11} + c_{11} \\
        z_7 &= g(z_6) & &= 26 z_{10} + d_{11} \\
        d_8 &= f(z_7, -12) & d_{12} &= f(z_{11}, -1) \\
        z_8 &= g(z_7) & z_{12} &= g(z_{11}) \\
        z_9 &= 26 z_8 + d_9 + c_9 & d_{13} &= f(z_{12}, 0) \\
        &= 26 z_8 + d_9 + 6 & z_{13} &= g(z_{12})  \\
        d_{10} &= f(z_9, -11) & d_{14} &= f(z_{13}, -11) \\
        z_{10} &= g(z_9) & z_{14} &= g(z_{13})
    }

    Thus, we are free to choose the digits $d_n$ for $n \in \braces{1, 2, 3, 5, 6, 9, 11}$ whereas those for $n \in \braces{4, 7, 8, 10, 12, 13, 14}$ are all determined based on those we choose.
    This massively reduces our required search space from $9^{14}$ possible model numbers to only $9^7$ model numbers, which is much more feasible to run in a reasonable amount of time.
    So, in our program, we search every possible combination of the appropriate digits, that is
    \gath{
        (d_1, d_2, d_3, d_5, d_6, d_9, d_{11}) \in D \times D \times D \times D \times D \times D \times D,
    }
    where $D = \braces{1, 2, 3, 4, 5, 6, 7, 8, 9}$ is the set of valid digits.
    For the first part we start each digit at $9$ then move backwards to $1$ since we are looking for the \emph{largest} valid model number, for the second part we do the opposite since we are looking for the \emph{smallest} model number.

    For each model number digits, we calculate $z_n$ and/or $d_n$ in sequence for increasing $n$ using the equations above.
    If any of the digits $d_n$ are not in the set $D$ we discard the model number, and similarly if we finally calculate $z_{14}$, and it is nonzero.
    Given our search order, the first model number that meets all of these conditions is our answer in either part.
    Finally, the first valid model digits are passed as input to the interpreted MONAD program to verify its validity.
}

\notes{2022}{11}{https://adventofcode.com/2022/day/11}{
    Implementing part two of this problem naively results in overflows even when using 64-bit integers.
    This is because:
    \begin{enumerate}
        \item We are no longer dividing the worry levels by 3 (floor division) at every inspection to reduce them.
        \item One of the monkeys is squaring the worry level.
        \item We need to compute this after $10000$ turns instead of the $20$ in part one.
    \end{enumerate}
    However, there is a trick we can use that will prevent these overflows without affecting the results.
    After each worry level is inspected, it is also tested by seeing whether it is divisible by some integer $m_k$ for monkey $k$, i.e. the test is whether
    \gath{
        x \equiv 0 \pmod{m_k},
    }
    where $x$ is the worry level.
    If there are $N$ monkeys, then let
    \gath{
        m = \prod_{k=1}^N m_k
    }
    be the product of these test modulo numbers.
    It is trivial to show that, if $x \equiv b \pmod{m_k}$ and $x \equiv a \pmod{m}$, then $a \equiv b \mod{m_k}$.
    This enables us to do all of our worry level arithmetic in modulo $m$ and have all the monkey's tests give the same results as though we were not restricting the arithmetic at all.
    The difference, of course, is that doing arithmetic modulo $m$ keeps the worry levels bounded in the interval $[0, m-1]$ instead of becoming arbitrarily large and overflowing our variables.
    This of course will change the actual worry levels being thrown around, but our answer is invariant to this change.
}


\end{document}
