\documentclass{article}

% For math environments
\usepackage{amsmath, amsfonts}
% For links
\usepackage[colorlinks=true,
    linkcolor = blue,
    urlcolor  = blue,
    citecolor = blue,
    anchorcolor = blue]{hyperref}
% So spaces are put between paragraphs
\usepackage{parskip}
% For figures
\usepackage{tikz}
% Set the margins to not be ridiculous
\usepackage[margin=0.75in]{geometry}
% For code blocks
\usepackage{listings}

% Make listings look better
\lstset{
  basicstyle=\ttfamily,
  columns=fullflexible,
  keepspaces=true,
}

% Useful commands
\newcommand\codebox[2]{
    \begin{center}
        \framebox[#1][l]{
            \lstinputlisting{#2}
        }
    \end{center}
}

% Monospace inline code
\def\code#1{\texttt{#1}}

% Commands that make life easier
\newcommand\gath[1]{\begin{gather} #1 \end{gather}}
\newcommand\gaths[1]{\begin{gather*} #1 \end{gather*}}
\newcommand\ali[1]{\begin{align} #1 \end{align}}
\newcommand\parens[1]{\left( #1 \right)}
\newcommand\squares[1]{\left[ #1 \right]}
\newcommand\braces[1]{\left\{ #1 \right\}}
\newcommand\angles[1]{\left\langle #1 \right\rangle}
\newcommand\deriv[2]{\frac{d #1}{d #2}}
\newcommand\abs[1]{\left| #1 \right|}
\newcommand\floor[1]{\left\lfloor #1 \right\rfloor}
\DeclareMathOperator{\lcm}{lcm}
\def\non{\nonumber \\}

% Commands for problems
\newcommand\notes[4]{
    \section*{#1 Day #2}
    \textbf{Puzzle:} \href{#3}{#3}

    #4
}


\begin{document}

\title{Advent of Code Solution Notes}
\author{Dan Whitman}
\date{}

\maketitle

\notes{2015}{25}{https://adventofcode.com/2015/day/25}{
    As part of this problem we have the following infinite table in which the order is filled in along diagonals:
    \begin{center}
        \begin{tabular}{c|ccccccc}
                     & 1        & 2        & 3        & 4        & 5        & 6        & $\cdots$ \\
            \hline
            1        & 1        & 3        & 6        & 10       & 15       & 21       & $\cdots$ \\
            2        & 2        & 5        & 9        & 14       & 20       & 27       & $\cdots$ \\
            3        & 4        & 8        & 13       & 19       & 26       & 34       & $\cdots$ \\
            4        & 7        & 12       & 18       & 25       & 33       & 42       & $\cdots$ \\
            5        & 11       & 17       & 24       & 32       & 41       & 51       & $\cdots$ \\
            6        & 16       & 23       & 31       & 40       & 50       & 61       & $\cdots$ \\
            $\vdots$ & $\vdots$ & $\vdots$ & $\vdots$ & $\vdots$ & $\vdots$ & $\vdots$ & $\ddots$ \\
        \end{tabular}
    \end{center}
    To solve the problem, for a given row and column number, we need to determine which number in the above table is at those coordinates.
    Given the diagonal pattern, this can be done analytically.
    Let $m$ denote a row and $n$ denote a column, where of course $n, m \geq 1$, and let $f(m, n)$ denote the number at the position.

    Now, we notice a pattern in the first row in which the $n$th number is the previous number plus $n$.
    This results in the following:
    \gath{
        f(1, n) = \sum_{j=1}^n j = \frac{n(n + 1)}{2},
    }
    which are the triangular numbers.
    The $n$th column also follows a similar pattern in which the $m$th number down is the previous number plus $n + m - 2$.
    This results in the following general formula we seek:
    \ali{
        f(m, n) &= f(1, n) + \sum_{i=0}^{m-2}(n + i) = f(1, n) + \sum_{i=1}^{m-1}(n + i - 1) \non
        &= \frac{n(n + 1)}{2} + \sum_{i=1}^{m-1} (n - 1) + \sum_{i=1}^{m-1} i = \frac{n(n + 1)}{2} + (m - 1)(n - 1) + \frac{m(m - 1)}{2} \non
        &= \frac{n(n + 1) + m(m - 1) + 2(n - 1)(m - 1)}{2} = \frac{n^2 + n + m^2 - m + 2nm - 2m - 2n + 2}{2} \non
        &= \frac{n^2 + m^2 + 2nm - n - 3m + 2}{2} \non
        &= \frac{(n + m)^2 - n -3m + 2}{2}.
    }
    Note that we have followed the usual convention that $\sum_{i=a}^b = 0$ when $b < a$, and also that clearly
    \gath{
        \sum_{i=1}^{m-1} i = \frac{m(m - 1)}{2} = 0
    }
    when $m = 1$ as we would like according to this convention.
}

\notes{2021}{8}{https://adventofcode.com/2021/day/8}{
    \def\WW{\mathbb{W}}
    \def\SS{\mathbb{S}}
    There are a total of seven segments, namely $\braces{a, b, c, d, e, f, g}$.
    For the digit $0 \leq n \leq 9$, let $S_n$ be the set of segments used to create the digit $n$, and let $\SS = \braces{S_n \mid 0 \leq n \leq 9}$ be the set of these digit sets.
    Therefore, from what we are given, we have
    \ali{
        S_0 &= \braces{a, b, c, e, f, g} & S_5 &= \braces{a, b, d, f, g} \non
        S_1 &= \braces{c, f} & S_6 &= \braces{a, b, d, e, f, g} \non
        S_2 &= \braces{a, c, d, e, g} & S_7 &= \braces{a, c, f} \non
        S_3 &= \braces{a, c, d, f, g} & S_8 &= \braces{a, b, c, d, e, f, g} \non
        S_4 &= \braces{b, c, d, f} & S_9 &= \braces{a, b, c, d, f, g} \nonumber
    }
    For a given entry, we must find the one-to-one map of the wire letters to the segment letters, as this allows us to determine which digits are being displayed.
    Note that this is only needed for part two of the problem.
    Denote these wire letters with $w_x$ where $x \in \braces{a, b, c, d, e, f, g}$ so that $w_x$ maps to segment $x$.
    For the entry, we are given a set of sets $\WW = \braces{W_n \mid 0 \leq n \leq 9}$, where each $W_n$ corresponds to $S_n$ and contains exactly those wires that correspond to the segments in $S_n$.
    So, for example, $W_7 = \braces{w_a, w_c, w_f}$.
    However, we initially do not know which $W_n \in \WW$ is which.

    The following algorithm was developed by experimentation with the sets in Python, and is implemented by the code to solve the problem.
    First, we \emph{do} know how many elements each $W_n \in \WW$ has, and as discussed in part one of the problem description, some sets $S_n \in \SS$ have a unique number of elements.
    In particular, $S_1$ is the only set with only two elements, $S_4$ is the only set with four elements, $S_7$ is the only set with three elements, and $S_8$ is the only set with all seven elements.
    All the other sets have either five or six elements.
    Hence, we know right away exactly which sets in $\WW$ are $W_1$, $W_4$, $W_7$, and $W_8$.

    Next, it was noticed that $S_7 - S_1 = \braces{a}$, and so
    \gath{
        \braces{w_a} = W_7 - W_1. \label{eqn:2021:8:wa}
    }
    Now note that $S_2$, $S_3$, and $S_5$ are the sets in $\SS$ with five elements.
    Similarly, $S_0$, $S_6$, and $S_9$ are the sets in $\SS$ with six elements.
    So let
    \gath{
        J_5 = \bigcap \braces{S_2, S_3, S_5} = \braces{a, d, g} \\
        J_6 = \bigcap \braces{S_0, S_6, S_9} = \braces{a, b, f, g}
    }
    and, analogously,
    \gath{
        I_5 = \bigcap \braces{W \in \WW \mid \abs{W} = 5} \\
        I_6 = \bigcap \braces{W \in \WW \mid \abs{W} = 6}
    }
    so that $I_5$ and $I_6$ are fully defined from our given entry data since we know how many elements each $W \in \WW$ has.
    It then follows that $J_5 \cap J_6 = \braces{a, g}$, and hence
    \gath{
        \braces{w_g} = \parens{I_5 \cap I_6} - \braces{w_a}
    }
    since we have already determined $w_a$ in \eqref{eqn:2021:8:wa}.
    Next we have that clearly $J_5 - \braces{a, g} = \braces{d}$ so that
    \gath{
        \braces{w_d} = I_5 - \braces{w_a, w_g}.
    }
    At this point we have determined three of the seven wire mappings.

    The next step is to notice that $J_6 \cap S_1 = \braces{f}$, and thus
    \gath{
        \braces{w_f} = I_6 \cap W_1.
    }
    From this it immediately follows that $S_1 - \braces{f} = \braces{c}$ so that
    \gath{
        \braces{w_c} = W_1 - \braces{w_f}.
    }
    The remaining two mappings are easily found at this point as we first have $S_4 - \braces{c, d, f} = \braces{b}$ so that
    \gath{
        \braces{w_b} = W_4 - \braces{w_c, w_d, w_f},
    }
    and $S_8 - \braces{a, b, c, d, f, g} = \braces{e}$ so that
    \gath{
        \braces{w_e} = W_8 - \braces{w_a, w_b, w_c, w_d, w_f, w_g}.
    }
    At this point we know the complete mapping from wires to segments and so can determine which digits are displayed.
}

\notes{2021}{17}{https://adventofcode.com/2021/day/17}{
    TODO
}

\notes{2021}{24}{https://adventofcode.com/2021/day/24}{
    This problem involved quite a bit of analytical work to reverse engineer the MONAD program (i.e. the puzzle input).
    This was necessary to determine the properties of valid model numbers since the search space of the $14$-digit model numbers is too large to exhaustively search.
    First, MONAD can be divided into $14$ sections that deal with each digit of the model number, and the sections are largely identical to each other except for some key parameters.
    The code for each section has the form:

    \codebox{3in}{2021-24-monad-template.txt}

    Each MONAD section has the three parameters $a_n$, $b_n$, and $c_n$, where we note that always $a_n \in \braces{1, 26}$.
    The actual parameters of each section are
    \begin{center}
        \begin{tabular}{c|ccc|c}
            $n$ & $a_n$ & $b_n$ & $c_n$ & Kind       \\
            \hline
            1   & 1     & 11    & 3     & Increasing \\
            2   & 1     & 14    & 7     & Increasing \\
            3   & 1     & 13    & 1     & Increasing \\
            4   & 26    & -4    & 6                  \\
            5   & 1     & 11    & 14    & Increasing \\
            6   & 1     & 10    & 7     & Increasing \\
            7   & 26    & -4    & 9                  \\
            8   & 26    & -12   & 9                  \\
            9   & 1     & 10    & 6     & Increasing \\
            10  & 26    & -11   & 4                  \\
            11  & 1     & 12    & 0     & Increasing \\
            12  & 26    & -1    & 7                  \\
            13  & 26    & 0     & 12                 \\
            14  & 26    & -11   & 1                  \\
        \end{tabular}
    \end{center}


    After the $n$th section let $z_n$ be the value in the \code{z} register at the end of each section, where of course initially $z_0 = 0$.
    Also let $d_n$ be the $n$th digit so that $1 \leq d_n \leq 9$.
    Then the above code for each section results in the following:
    \gath{
        z_n = \begin{cases}
            \floor{\frac{z_{n-1}}{a_n}}                & d_n = (z_{n-1} \mod 26) + b_n    \\
            26 \floor{\frac{z_{n-1}}{a_n}} + d_n + c_n & d_n \neq (z_{n-1} \mod 26) + b_n \\
        \end{cases} \label{eqn:2021:24:f}
    }
    First, note that, for all the sections for which $a_n = 1$, we have that $b_n > 9$ so that clearly $(z_{n-1} \mod 26) + b_n > 9$, and hence $d_n \neq (z_{n-1} \mod 26) + b_n$.
    From this it follows that
    \gath{
        z_n = 26 \floor{\frac{z_{n-1}}{a_n}} + d_n + c_n = 26 z_{n-1} + d_n + c_n
    }
    since $a_n = 1$.
    It will obviously be the case here that $z_n > z_{n-1}$ and, for this reason, these sections are marked as ``Increasing'' in the table above.

    Since a valid model number must result in $z_{14} = 0$ and half of the sections will increase $z$, let us assume that the other seven sections must \emph{decrease} $z$.
    In particular, we suppose that all of these sections (viz. for $n \in \braces{4, 7, 8, 10, 12, 13, 14}$) apply the first case of \eqref{eqn:2021:24:f}, which imposes restrictions on these digits.
    Now, the first three sections result in the following:
    \ali{
        z_1 &= 26 z_0 + d_1 + c_1 \non
        &= d_1 + 3 \\
        z_2 &= 26 z_1 + d_2 + c_2 = 26(d_1 + 3) + d_2 + c_2 \non
        &= 26 d_1 + d_2 + 85 \\
        z_3 &= 26 z_2 + d_3 + c_3 = 26(26 d_1 + d_2 + 85) + d_3 + 1 \non
        &= 676d_1 + 26d_2 + d_3 + 2211.
    }
    Regarding $d_4$ and $z_4$, define functions
    \ali{
        f(z, b) &= (z \mod 26) + b \\
        g(z) &= \floor{\frac{z}{26}}
    }
    for brevity.
    Then, since we assume that $z_4$ is determined from the first case of \eqref{eqn:2021:24:f}, it must be that
    \ali{
        d_4 = f(z_3, -4) \\
        z_4 = g(z_3).
    }
    Since the $n \in \braces{5, 6}$ sections again increase $z$, we have
    \ali{
        z_5 &= 26 z_4 + d_5 + c_5 \non
        &= 26 z_4 + d_5 + 14 \\
        z_6 &= 26 z_5 + d_6 + c_6 = 26(26 z_4 + d_5 + 14) + d_6 + 7 \non
        &= 676 z_4 + 26 d_5 + d_6 + 371.
    }
    Continuing in this way for both the increasing and decreasing sections gives
    \ali{
        d_7 &= f(z_6, -4) & z_{11} &= 26 z_{10} + d_{11} + c_{11} \\
        z_7 &= g(z_6) & &= 26 z_{10} + d_{11} \\
        d_8 &= f(z_7, -12) & d_{12} &= f(z_{11}, -1) \\
        z_8 &= g(z_7) & z_{12} &= g(z_{11}) \\
        z_9 &= 26 z_8 + d_9 + c_9 & d_{13} &= f(z_{12}, 0) \\
        &= 26 z_8 + d_9 + 6 & z_{13} &= g(z_{12})  \\
        d_{10} &= f(z_9, -11) & d_{14} &= f(z_{13}, -11) \\
        z_{10} &= g(z_9) & z_{14} &= g(z_{13})
    }

    Thus, we are free to choose the digits $d_n$ for $n \in \braces{1, 2, 3, 5, 6, 9, 11}$ whereas those for $n \in \braces{4, 7, 8, 10, 12, 13, 14}$ are all determined based on those we choose.
    This massively reduces our required search space from $9^{14}$ possible model numbers to only $9^7$ model numbers, which is much more feasible to run in a reasonable amount of time.
    So, in our program, we search every possible combination of the appropriate digits, that is
    \gath{
        (d_1, d_2, d_3, d_5, d_6, d_9, d_{11}) \in D \times D \times D \times D \times D \times D \times D,
    }
    where $D = \braces{1, 2, 3, 4, 5, 6, 7, 8, 9}$ is the set of valid digits.
    For the first part we start each digit at $9$ then move backwards to $1$ since we are looking for the \emph{largest} valid model number, for the second part we do the opposite since we are looking for the \emph{smallest} model number.

    For each model number digits, we calculate $z_n$ and/or $d_n$ in sequence for increasing $n$ using the equations above.
    If any of the digits $d_n$ are not in the set $D$ we discard the model number, and similarly if we finally calculate $z_{14}$, and it is nonzero.
    Given our search order, the first model number that meets all of these conditions is our answer in either part.
    Finally, the first valid model digits are passed as input to the interpreted MONAD program to verify its validity.
}

\end{document}
